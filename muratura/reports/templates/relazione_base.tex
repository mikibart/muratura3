% =============================================================================
% RELAZIONE DI CALCOLO STRUTTURALE - MURATURA 3.0
% Template LaTeX per edifici in muratura conforme NTC 2018
% =============================================================================

\documentclass[a4paper,11pt,twoside]{report}

% -----------------------------------------------------------------------------
% PACKAGES
% -----------------------------------------------------------------------------
\usepackage[utf8]{inputenc}
\usepackage[T1]{fontenc}
\usepackage[italian]{babel}
\usepackage{geometry}
\usepackage{graphicx}
\usepackage{booktabs}
\usepackage{longtable}
\usepackage{array}
\usepackage{multirow}
\usepackage{xcolor}
\usepackage{colortbl}
\usepackage{fancyhdr}
\usepackage{titlesec}
\usepackage{tocloft}
\usepackage{hyperref}
\usepackage{amsmath}
\usepackage{amssymb}
\usepackage{siunitx}
\usepackage{enumitem}
\usepackage{float}
\usepackage{caption}
\usepackage{subcaption}
\usepackage{pdfpages}
\usepackage{lastpage}

% -----------------------------------------------------------------------------
% GEOMETRY
% -----------------------------------------------------------------------------
\geometry{
    a4paper,
    left=25mm,
    right=20mm,
    top=25mm,
    bottom=25mm,
    headheight=15pt
}

% -----------------------------------------------------------------------------
% COLORS
% -----------------------------------------------------------------------------
\definecolor{primary}{RGB}{26,82,118}
\definecolor{secondary}{RGB}{46,134,171}
\definecolor{lightgray}{RGB}{248,249,250}
\definecolor{darkgray}{RGB}{51,51,51}
\definecolor{success}{RGB}{40,167,69}
\definecolor{warning}{RGB}{255,193,7}
\definecolor{danger}{RGB}{220,53,69}

% -----------------------------------------------------------------------------
% HEADER/FOOTER
% -----------------------------------------------------------------------------
\pagestyle{fancy}
\fancyhf{}
\fancyhead[LE,RO]{\thepage}
\fancyhead[LO]{\nouppercase{\leftmark}}
\fancyhead[RE]{<<project_name>>}
\fancyfoot[C]{\footnotesize Relazione generata con MURATURA 3.0}
\renewcommand{\headrulewidth}{0.4pt}
\renewcommand{\footrulewidth}{0.2pt}

% -----------------------------------------------------------------------------
% TITLE FORMATTING
% -----------------------------------------------------------------------------
\titleformat{\chapter}[display]
    {\normalfont\huge\bfseries\color{primary}}
    {\chaptertitlename\ \thechapter}{20pt}{\Huge}
\titleformat{\section}
    {\normalfont\Large\bfseries\color{secondary}}
    {\thesection}{1em}{}
\titleformat{\subsection}
    {\normalfont\large\bfseries\color{darkgray}}
    {\thesubsection}{1em}{}

% -----------------------------------------------------------------------------
% TABLE SETTINGS
% -----------------------------------------------------------------------------
\newcolumntype{L}[1]{>{\raggedright\arraybackslash}p{#1}}
\newcolumntype{C}[1]{>{\centering\arraybackslash}p{#1}}
\newcolumntype{R}[1]{>{\raggedleft\arraybackslash}p{#1}}

\newcommand{\tableheader}[1]{\cellcolor{primary}\textcolor{white}{\textbf{#1}}}

% -----------------------------------------------------------------------------
% CUSTOM COMMANDS
% -----------------------------------------------------------------------------
\newcommand{\dcrok}[1]{\textcolor{success}{\textbf{#1}}}
\newcommand{\dcrwarn}[1]{\textcolor{warning}{\textbf{#1}}}
\newcommand{\dcrfail}[1]{\textcolor{danger}{\textbf{#1}}}

\newcommand{\formula}[2]{%
    \begin{tcolorbox}[colback=lightgray,colframe=secondary,boxrule=0.5pt,arc=2pt,left=5pt,right=5pt,top=5pt,bottom=5pt]
        #1 \hfill \textit{#2}
    \end{tcolorbox}
}

% SI Units setup
\sisetup{
    output-decimal-marker = {,},
    group-separator = {.},
    per-mode = symbol
}

% Hyperref setup
\hypersetup{
    colorlinks=true,
    linkcolor=primary,
    filecolor=primary,
    urlcolor=secondary,
    pdftitle={Relazione di Calcolo - <<project_name>>},
    pdfauthor={MURATURA 3.0},
    pdfsubject={Analisi strutturale edificio in muratura NTC 2018}
}

% =============================================================================
% DOCUMENT
% =============================================================================
\begin{document}

% -----------------------------------------------------------------------------
% TITLE PAGE
% -----------------------------------------------------------------------------
\begin{titlepage}
    \centering
    \vspace*{2cm}

    {\huge\bfseries\color{primary} RELAZIONE DI CALCOLO\\STRUTTURALE\par}
    \vspace{1.5cm}

    {\LARGE\color{secondary} <<project_name>>\par}
    \vspace{0.5cm}

    {\large Analisi strutturale edificio in muratura\\conforme NTC 2018\par}
    \vspace{3cm}

    \begin{tabular}{@{}ll@{}}
        \textbf{Committente:} & <<committente>>\\[5pt]
        \textbf{Progettista:} & <<progettista>>\\[5pt]
        \textbf{Ubicazione:} & <<ubicazione>>\\[5pt]
        \textbf{Data:} & <<data>>\\[5pt]
        \textbf{Codice Progetto:} & <<codice_progetto>>\\
    \end{tabular}

    \vfill

    {\footnotesize Relazione generata con MURATURA 3.0\\
    Software per analisi strutturale edifici in muratura\\
    Conforme a NTC 2018 e Circolare n. 7/2019\par}

\end{titlepage}

% -----------------------------------------------------------------------------
% TABLE OF CONTENTS
% -----------------------------------------------------------------------------
\tableofcontents
\clearpage

% =============================================================================
% CHAPTER 1: PREMESSA
% =============================================================================
\chapter{Premessa}

La presente relazione di calcolo strutturale è redatta ai sensi delle Norme Tecniche per le Costruzioni (NTC 2018 - D.M. 17 gennaio 2018) e della relativa Circolare esplicativa n. 7 del 21 gennaio 2019.

Oggetto dell'analisi è l'edificio in muratura portante denominato \textbf{``<<project_name>>''}, ubicato in <<ubicazione>>.

L'analisi strutturale è stata condotta utilizzando il software MURATURA 3.0, conforme alle prescrizioni normative vigenti per la verifica di edifici in muratura.

% =============================================================================
% CHAPTER 2: DESCRIZIONE OPERA
% =============================================================================
\chapter{Descrizione dell'Opera}

\section{Generalità}

\begin{table}[H]
    \centering
    \begin{tabular}{@{}ll@{}}
        \toprule
        \tableheader{Parametro} & \tableheader{Valore}\\
        \midrule
        Tipologia edificio & <<tipologia_edificio>>\\
        Tipo costruzione & <<tipo_costruzione>>\\
        Categoria intervento & <<categoria_intervento>>\\
        Numero piani fuori terra & <<n_piani>>\\
        Numero piani interrati & <<n_piani_interrati>>\\
        Altezza massima & \SI{<<altezza_max>>}{\meter}\\
        Superficie coperta & \SI{<<superficie_coperta>>}{\meter\squared}\\
        \bottomrule
    \end{tabular}
    \caption{Caratteristiche generali dell'edificio}
\end{table}

\section{Vita Nominale e Classe d'Uso}

Ai sensi del \S2.4 delle NTC 2018:

\begin{table}[H]
    \centering
    \begin{tabular}{@{}llc@{}}
        \toprule
        \tableheader{Parametro} & \tableheader{Valore} & \tableheader{Riferimento}\\
        \midrule
        Vita nominale $V_N$ & \SI{<<VN>>}{anni} & Tab. 2.4.I\\
        Classe d'uso & <<classe_uso>> & Tab. 2.4.II\\
        Coefficiente d'uso $C_U$ & <<CU>> & Tab. 2.4.II\\
        Periodo di riferimento $V_R$ & \SI{<<VR>>}{anni} & $V_R = V_N \times C_U$\\
        \bottomrule
    \end{tabular}
    \caption{Parametri di vita nominale e classe d'uso}
\end{table}

% =============================================================================
% CHAPTER 3: NORMATIVE
% =============================================================================
\chapter{Normative di Riferimento}

\begin{itemize}[leftmargin=*]
    \item \textbf{D.M. 17/01/2018} - Aggiornamento delle ``Norme Tecniche per le Costruzioni''
    \item \textbf{Circolare 21/01/2019 n. 7} - Istruzioni per l'applicazione delle NTC 2018
    \item \textbf{Eurocodice 6} - Progettazione delle strutture in muratura
    \item \textbf{Eurocodice 8} - Progettazione delle strutture per la resistenza sismica
    \item \textbf{O.P.C.M. 3274/2003} e s.m.i. - Norme tecniche per le costruzioni in zona sismica
\end{itemize}

% =============================================================================
% CHAPTER 4: MATERIALI
% =============================================================================
\chapter{Caratteristiche dei Materiali}

\section{Muratura}

Le caratteristiche meccaniche della muratura sono state determinate secondo la Tabella C8.5.I della Circolare n. 7/2019, con applicazione dei coefficienti correttivi previsti.

\begin{table}[H]
    \centering
    \begin{tabular}{@{}lccr@{}}
        \toprule
        \tableheader{Parametro} & \tableheader{Simbolo} & \tableheader{Valore} & \tableheader{Unità}\\
        \midrule
        <<materiali_tabella>>
        \bottomrule
    \end{tabular}
    \caption{Proprietà meccaniche della muratura}
\end{table}

\section{Livello di Conoscenza}

\begin{tcolorbox}[colback=blue!5,colframe=primary,title=Livello di Conoscenza]
    \textbf{Livello:} <<livello_conoscenza>>\\
    \textbf{Fattore di confidenza:} $FC = <<FC>>$
\end{tcolorbox}

% =============================================================================
% CHAPTER 5: CARICHI
% =============================================================================
\chapter{Analisi dei Carichi}

\section{Carichi Permanenti Strutturali (G1)}

<<carichi_g1>>

\section{Carichi Permanenti Non Strutturali (G2)}

<<carichi_g2>>

\section{Carichi Variabili (Q)}

<<carichi_q>>

\section{Combinazioni di Carico}

Le combinazioni di carico sono state definite secondo il \S2.5.3 delle NTC 2018.

% =============================================================================
% CHAPTER 6: AZIONE SISMICA
% =============================================================================
\chapter{Azione Sismica}

\section{Parametri di Pericolosità}

\begin{table}[H]
    \centering
    \begin{tabular}{@{}ccccc@{}}
        \toprule
        \tableheader{Stato Limite} & \tableheader{$T_R$ [anni]} & \tableheader{$a_g$ [g]} & \tableheader{$F_0$} & \tableheader{$T_C^*$ [s]}\\
        \midrule
        <<parametri_sismici>>
        \bottomrule
    \end{tabular}
    \caption{Parametri di pericolosità sismica}
\end{table}

\section{Categoria Sottosuolo e Topografica}

\begin{table}[H]
    \centering
    \begin{tabular}{@{}lcc@{}}
        \toprule
        \tableheader{Parametro} & \tableheader{Categoria} & \tableheader{Coefficiente}\\
        \midrule
        Sottosuolo & <<cat_sottosuolo>> & $S_S = <<SS>>$\\
        Topografia & <<cat_topografia>> & $S_T = <<ST>>$\\
        Amplificazione totale & --- & $\mathbf{S = <<S>>}$\\
        \bottomrule
    \end{tabular}
    \caption{Categorie di sottosuolo e topografia}
\end{table}

\section{Spettro di Risposta Elastico}

\begin{figure}[H]
    \centering
    <<spettro_immagine>>
    \caption{Spettro di risposta elastico SLV}
\end{figure}

\section{Fattore di Struttura}

\begin{equation}
    q = q_0 \times K_R = <<q0>> \times <<KR>> = \mathbf{<<q>>}
\end{equation}

secondo le indicazioni di NTC \S7.8.1.3.

% =============================================================================
% CHAPTER 7: MODELLO
% =============================================================================
\chapter{Modellazione Strutturale}

\section{Telaio Equivalente}

L'edificio è stato modellato mediante telaio equivalente secondo le indicazioni della Circolare \S C8.7.1. Il modello identifica automaticamente maschi murari (piers) e fasce di piano (spandrels).

\begin{table}[H]
    \centering
    \begin{tabular}{@{}lr@{}}
        \toprule
        \tableheader{Elemento} & \tableheader{Quantità}\\
        \midrule
        Maschi murari (piers) & <<n_maschi>>\\
        Fasce di piano (spandrels) & <<n_fasce>>\\
        Nodi rigidi & <<n_nodi>>\\
        \bottomrule
    \end{tabular}
    \caption{Elementi del telaio equivalente}
\end{table}

\section{Rigidezza di Piano}

<<rigidezza_piano>>

% =============================================================================
% CHAPTER 8: ANALISI
% =============================================================================
\chapter{Analisi Strutturale}

\section{Metodi di Analisi}

L'analisi strutturale è stata condotta con i seguenti metodi:

<<metodi_analisi>>

\section{Analisi Modale}

<<analisi_modale>>

\section{Analisi Pushover}

<<analisi_pushover>>

% =============================================================================
% CHAPTER 9: VERIFICHE
% =============================================================================
\chapter{Verifiche di Sicurezza}

\section{Verifiche SLU - Maschi Murari}

<<verifiche_maschi>>

\section{Verifiche SLU - Fasce}

<<verifiche_fasce>>

\section{Verifiche SLD}

<<verifiche_sld>>

\section{Indice di Rischio Sismico}

\begin{tcolorbox}[colback=green!5,colframe=success]
    \textbf{Indice di Rischio (IR):} <<IR>>\\
    \textbf{Classe di Rischio Sismico:} <<classe_rischio>>
\end{tcolorbox}

% =============================================================================
% CHAPTER 10: MECCANISMI LOCALI
% =============================================================================
\chapter{Meccanismi Locali}

<<meccanismi_locali>>

% =============================================================================
% CHAPTER 11: RINFORZI
% =============================================================================
\chapter{Interventi di Rinforzo}

<<interventi_rinforzo>>

% =============================================================================
% CHAPTER 12: CONCLUSIONI
% =============================================================================
\chapter{Conclusioni}

<<conclusioni>>

% =============================================================================
% APPENDICES
% =============================================================================
\appendix

\chapter{Allegati Grafici}

% Qui si possono includere le tavole

\chapter{Tabulati di Calcolo}

% Qui si possono includere i tabulati dettagliati

% =============================================================================
\end{document}
